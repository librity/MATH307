\input ../../../texinputs/latexdefs
\standardpackages
\usepackage[utf8]{inputenc}
\parindent = 0pt

\begin{document}

\begin{center}
{\normalfont\Large\bfseries Math 307 Homework 1  due  Oct 14 }
\end{center}
\bigskip\bigskip

\section{Norms and Condition Numbers}

When $1\le p < \infty$, we defined the $p$ norm for vectors $\xx = [x_1, x_2, \ldots, x_n]$
to be
$$
\|\xx\|_p = \left(\sum_{i=1}^n |x_i|^p\right)^{1/p}.
$$
When $p=\infty$ we define
$$
\|\xx\|_\infty =\max\{|x_1|,\ldots, |x_n|\}
$$
\bigskip
\hrule
\bigskip
1. Show  that $\|\xx\|_\infty = \lim_{p\rightarrow\infty}\|\xx\|_p$.

2. Use the Cauchy Schwarz inequality
$$
|\yy^T \xx| \le \|\xx\|_2\|\yy\|_2
$$
to show that 
$$
\|\xx\|_2 = \max_{\yy\ne 0}\frac{ |\yy^T \xx|}{\|\yy\|_2}
$$


\bigskip
\hrule
\bigskip

The generalization of the Cauchy-Schwarz inequality to $p$ norms is called the 
Hölder inequality. 


 To state the general case, define the conjugate exponent $q$ to $p$ so that $$\frac{1}{p} + \frac{1}{q} = 1.$$
 
 Then Hölder's inequality states

$$
|\yy^T \xx| \le \|\xx\|_p\|\yy\|_q
$$

\bigskip
\bigskip
\hrule
\bigskip

\vfill\eject

3. Prove the special case

$$
|\yy^T \xx| \le \|\xx\|_\infty\|\yy\|_1
$$

 
\hrule
\bigskip
4. Suppose you are not sure if this is true, and before sinking a lot of time into proving it, you decide to test the inequality on a bunch of random vectors in $\mathbb R^5$ using python.
Show how you would do that for $p=3$.

\bigskip
\hrule
\bigskip

We defined the matrix norm to be
$$
\|A\| = \max_{\xx\ne 0} \frac{\|A\xx\|}{\|\xx\|}.
$$

\bigskip
\hrule
\bigskip
5. Show that for a diagonal matrix  $A=\begin{bmatrix}\lambda_1&0&0\\0&\lambda_2&0\\0&0&\lambda_3\\\end{bmatrix}$ the matrix norm is equal to
$$
\|A\| = \max\{|\lambda_1|,|\lambda_2|,|\lambda_3|\}
$$
\bigskip
Based on this resutl you might guess that 
$$
\|A\| = \max\{  | \lambda|  :  \lambda \in \hbox{eigenvalues}(A)\}
$$
6.  Use python to show that this is not correct, but might be for {\it real, symmetric} matrices $A$
\bigskip
\hrule
\bigskip
We will see later in the course that for a real matrix $A$, $\|A\|^2$ is the largest eigenvalue of $A^TA$. 
\bigskip
\hrule
\bigskip
7. Use this to find a formula  for the norm of any $2\times 2$ matrix. 
\bigskip\bigskip
\hrule
\bigskip


\section{Balancing chemical reactions}

8. Read section 1.6 in  Keith Nicholson's text and do problems 1.6.3 and 1.6.4.
\bigskip

\section{Jupyter notebook problems}
9--14. There are 3 Jupyter notebooks with problems: Do these five problems in the notebooks, highlighting your solutions. 
\bigskip

a. homework1LU.ipynb  (2 problems)
\bigskip

b. homework1\_chebyshev.ipynb (1 long answer problem)
\bigskip

c. homework1\_BVP.ipynb together with 3 files d.png, d2.png and d3.png to be placed in a subdirectory called img  (2 problems)

\end{document}
